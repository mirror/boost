
% Introduction/Motivation, etc. (Dave & Jeremy)

% iterator policies   (Dave)
%   default policies
%   type<> wrapper
% \cite{alexandrescu01:_modern_cpp_design}

% iterator_comparisons base (B&N) (Jeremy) \cite{Barton94}
%  workaround for g++ compiler bug with friend functions?

% operator_array_proxy (Dave)

% default pointer/reference type selection (Dave)

% wrapping non-iterators (count_iterator) (Jeremy)

% named template parameters (Jeremy)

% operator[] return type (Dave)

% the static asserts (Dave)

% generators (Jeremy)
%   type generators
%     tempting to try to use inheritance to replace
%        templated typedef, but that doesn't work.
%   object generators

% const/non-const interoperability (Dave)

% implementing const/mutable iterators with the same class
%   common mistake is to make the return type of operator*()
%     depend on whether the iterator object is const/non-const.
%     See the transform iterator in \cite{TMPW00:Weiser}

% custom iterators \cite{TMPW00:Baus}
%   generating iterators 
%     line iterator \cite{austern99:_gener_progr_stl}
%     constant iterator \cite{koenig97:_rumin_cpp}

% reverse iterator, front_insert_iterator, back_insert_iterator,
% insert_iterator \cite{iso98:_cpp_final_draft_standard}

% view iterators
%   \cite{TMPW00:Weiser}

% future work, container adaptors

\documentclass{netobjectdays}


\usepackage{times}

\newif\ifpdf
\ifx\pdfoutput\undefined
   \pdffalse
\else
   \pdfoutput=1
   \pdftrue
\fi

\ifpdf
  \usepackage[
              pdftex,
              colorlinks=true,
              linkcolor=blue,filecolor=blue,pagecolor=blue,urlcolor=blue
              ]{hyperref}
\fi

\ifpdf
  \newcommand{\concept}[1]{\hyperref[concept:#1]{\textsf{#1}}}
  \newcommand{\stlconcept}[1]{\href{http://www.sgi.com/Technology/STL/#1.html}{\textsf{#1}}}
  \newcommand{\link}[2]{\hyperref[#1]{#2}}
\else
  \newcommand{\concept}[1]{\textsf{#1}}
  \newcommand{\stlconcept}[1]{\textsf{#1}}
  \newcommand{\href}[2]{#2}
  \newcommand{\link}[2]{#2}
\fi

\newcommand{\code}[1]{{\small \texttt{#1}}}

\newcommand{\Note}[1]{\marginpar{\begin{flushleft}%
  {%%\tiny %%\footnotesize
  {\bf Note:}  #1}%
\end{flushleft}}}


\begin{document}

\title{Generating Iterator Types}

\author{David Abrahams$^\dag$ and Jeremy Siek$^\ddag$ \\
\\
$^\dag$ Altra Broadband \\
\texttt{abrahams@altrabroadband.com}\\
\\
$^\ddag$ Computer Science Department \\
Indiana University \\
Lindley Hall \\
150 S. Woodlawn Ave. \\
Bloomington, IN\ \ 47405-7104\\
\texttt{jsiek@cs.indiana.edu}
}

\maketitle


\begin{abstract}
The iterator abstraction is one of the most commonly used in
programming and a considerable amount of time is spent building new
iterator types. However, implementing an iterator type that satisfies
the C++ Standard requirements for an iterator can be
challenging. There are a number of common mistakes that people make,
and there are necessary complexities in a C++ Standard conforming
implementation that one would rather not have to think about. In this
paper we present the iterator type generator in the Boost Iterator
Adaptor Library. This generator simplifies the creation of iterators;
it automates the error-prone and redundant parts of the implementation
and greatly simplifies the creation of iterator types that are
variations on other iterators (creating iterator adaptors).  The
design of the Iterator Adaptor Library is an example of policy-based
design, and the implementation employs template meta-programming.
\end{abstract}


\section{Introduction}

%- defaults make it easy to adapt an iterator
%- extensions from base operations to other operations make it
%  easier to create iterators

Iterators play an important role in modern C++ programing. The
iterator is the central abstraction of the algorithms of the Standard
Library and creating new iterator types and adapting old ones are
common tasks for C++ programmers. There are plenty of examples of
custom-made iterators in the literature: the
\code{line\_iterator}~\cite{austern99:_gener_progr_stl},
\code{Constant\_iterator}~\cite{koenig97:_rumin_cpp}, 
\code{istream\_iterator} and
\code{ostream\_iterator}~\cite{iso98:_cpp_final_draft_standard} to
name a few. Also a growing number of generic iterator adaptors are
available: \code{Checked\_iter}~\cite{stroustrup00:_cpp_prog_lang},
iterators of the View Template Library~\cite{TMPW00:Weiser}, custom
and smart iterators~\cite{becker98:_smart_iteraters,TMPW00:Baus},
compound iterators~\cite{alexandrescu98:_compound_iters}, and several
iterators in the MTL~\cite{siek99:_scitools}.

For an iterator to be usable with the Standard algorithms (and other
generic algorithms in third-party libraries), it must fulfill the
Standard requirements for an iterator type, which range from the few
requirements of an \stlconcept{InputIterator} to the many requirements
of a \stlconcept{RandomAccessIterator}. Implementing an iterator class
that meets these requirements is a tedious and error-prone task
despite the fact that most iterators are conceptually simple.

\subsection{Redundant Operators}

Perhaps the most obvious of reasons that implementing an iterator can
be tediuos is that there are lots of redundant operators.  That is,
there are many operators that can be trivially defined in terms of
other operators. For example, the \code{operator++(int)} is often best
implemented in terms of \code{operator++()} as the example below
shows.

{\footnotesize
\begin{verbatim}
  class iter {
    // ...
    iter& operator++() { /* ... */ return *this; }
    iter operator++(int) { iter tmp(*this); ++*this; return tmp; }
  };
\end{verbatim}
}

For a full \stlconcept{RandomAccessIterator}, there are a total of 17
operators. 7 of the operators are fundamental while the other 10 are
redundant.

% 7 core operations
% 10 derived operations
 
% \code{operator->}
% \code{operator[]}
% \code{operator++(int)},
% \code{operator--(int)},
% \code{operator-=},
% \code{operator+},
% \code{operator!=},
% \code{operator>}, 
% \code{operator<=},
% \code{operator>=}


\subsection{Delagation of Operators for Iterator Adaptors}

It is often the case that an iterator adaptor changes the meaning of
one or two operators while leaving the rest of the operators defined
in the same way as the underlying iterator.  This is typically
implemented with delegating functions. The following example shows an
excerpt from an \code{indirect\_iterator} adaptor, which takes an
iterator over pointers and creates an iterator over the things pointed
to. The \code{operator*} is changed to dereference twice but all the
other operators stay the same. Writing all of the delegating functions
for the \code{indirect\_iterator} would be a tedious task.

{\footnotesize
\begin{verbatim}
  template <class Iterator> class indirect_iterator {
  public:
    // Adapt the meaning of dereference
    reference operator*() const {
      return **iter; // dereference twice
    }
    // Delegating the implementation to the underlying iterator.
    iter_adaptor& operator++() { ++iter; return *this; }
    iter_adaptor& operator--() { --iter; return *this; }
    // delegate for all the other operators...
  private:
    Iterator iter;
  };
\end{verbatim}
}

\subsection{Iterator Complexities}

In addition to the tedious aspects of iterator implementation, there
are some complexities that trip up even the most experienced of
programmers.

\subsubsection{Constant/Mutable Iterator Interactions}

One of the main situations in which iterators are used is inside
containers. These iterators usually come in pairs: a constant iterator
type and a mutable iterator type. It is desirable to allow the
constant and mutable iterators to interoperate in terms of their
binary operators. For example, suppose that you are implementing a
container type \code{C}. Then you ought to define the following four
version of \code{operator==}.

{\footnotesize
\begin{verbatim}
  bool operator==(const C::iterator& x, const C::iterator& y);
  bool operator==(const C::const_iterator& x, const C::iterator& y);
  bool operator==(const C::iterator& x, const C::const_iterator& y);
  bool operator==(const C::const_iterator& x, const C::const_iterator& y);
\end{verbatim}
}

Implementers often forget to define the operators for const/mutable
iterator interaction. In addition, iterator adaptors applied to these
kinds of iterators should propagate the ability to interact. For
example, a reverse iterator adaptor applied to \code{C::iterator} and
\code{C::const\_iterator} should result in reverse iterator types that
also have operators defined for the const/mutable interactions.


\subsubsection{Constant/Mutable Iterator Implementation}

Another subtlety in the implementation of iterators is how the the
distinction between constant and mutable iterators affects the
implementation. It is obvious that a constant iterator should have a
const \code{reference} type, while a mutable iterator should have a
non-const \code{reference}, though in other regards the constant and
mutable versions of an iterator are the same. It is therefore
desirable to implement both versions of the iterator with a single
class. It is possible to do this, however some care must be taken.
One common mistake is that the programmer will confuse the difference
between a const iterator object and a constant iterator. Such a
misunderstanding can, for example, lead to an iterator class that has
two versions of \code{operator*}, one that is a const member function
and one that is not.
 
{\footnotesize
\begin{verbatim}
  // this is a mistake
  reference operator*();
  const_reference operator*() const;
\end{verbatim}
}

The right way to implement both a constant and mutable iterators using
the same class is to make the iterator a class template and make the
reference type a parameter. To create the constant iterator a const
reference would be used as the template argument and to create the
mutable iterator a non-const reference would be used as the template
argument. There should be only one \code{operator*} that returns the
\code{reference} type and the member function should be const since
dereferencing an iterator does not change the state of the iterator
object itself (unlike \code{operator++}).

{\footnotesize
\begin{verbatim}
  // this is right
  reference operator*() const;
\end{verbatim}
}


\subsubsection{Input Iterators and \code{operator->}}

When creating an iterator adaptor that can accept an
\stlconcept{InputIterator} as the adapted type some extra care must be
taken in the implementation of \code{operator->}. \Note{Dave fills in
the rest}



% Automatic implementation of redundant operators
% Default delegation to adapted iterator

% complexities: 
%   const-non const interaction
%   const/mutable iterator distinction
%   input iterator \code{operator->}

\section{The Boost \code{iterator\_adaptor}}



\subsection{Example}

It is often useful to automatically apply some function to the value
returned by dereferencing an iterator. The transform iterator of the
Iterator Adaptor Library makes it easy to create an iterator adaptor
which does just that. Here we will show how easy it is to implement
the transform iterator using the
\code{iterator\_adaptor} template.

We want to be able to adapt a range of iterators and functions, so the
policies class will have a template parameter for the function type
and it will have a data member of that type. We know that the function
takes one argument and that we'll need to be able to deduce the
\code{result\_type} of the function so we can use it for the adapted
iterator's \code{value\_type}. \stlconcept{AdaptableUnaryFunction} is
the \textsf{concept}\cite{austern99:_gener_progr_stl} that fulfills
those requirements.

To implement a transform iterator we will only change one of the base
iterator's behaviors, so the \code{transform\_iterator\_policies}
class can inherit the rest from \code{default\_iterator\_policies}. We
will define the \code{dereference()} member function, which is used
to implement \code{operator*()} of the adapted iterator. The
implementation will dereference the base iterator and apply the
function object. The \code{type<Reference>} parameter is used
to convey the appropriate return type. The complete code for
\code{transform\_iterator\_policies} is:

{\footnotesize
\begin{verbatim}
  template <class AdaptableUnaryFunction>
  struct transform_iterator_policies : public default_iterator_policies
  {
    transform_iterator_policies() { }

    transform_iterator_policies(const AdaptableUnaryFunction& f)
      : m_f(f) { }

    template <class Reference, class BaseIterator>
    Reference dereference(type<Reference>, const BaseIterator& i) const
      { return m_f(*i); }

    AdaptableUnaryFunction m_f;
  };
\end{verbatim}
}

The next step is to use the \code{iterator\_adaptor} template to
construct the transform iterator type. The nicest way to package the
construction of the transform iterator is to create a \emph{type
generator}, which is a class template whose sole purpose is to
simplify the instantiation of some other complicated class
template. It fulfills the same need as a templated typedef would if
that were part of the {C++} language.

The first template parameter to the generator will be the type of the
function object and the second will be the base iterator type. We use
\code{iterator\_adaptor} to define the transform iterator type as a
nested \code{typedef} inside the
\code{transform\_iterator\_generator} class.  Because the function may
return by-value, we must limit the \code{iterator\_category} to 
\stlconcept{InputIterator}, and the iterator's \code{reference} type cannot be a
true reference (the standard allows this for input iterators), so in
this case we can use few of \code{iterator\_adaptor}'s default template
arguments.

{\footnotesize
\begin{verbatim}
  template <class AdaptableUnaryFunction, class Iterator>
  struct transform_iterator_generator
  {
      typedef typename AdaptableUnaryFunction::result_type value_type;
  public:
      typedef iterator_adaptor<Iterator, 
          transform_iterator_policies<AdaptableUnaryFunction>,
          value_type, value_type, value_type*, std::input_iterator_tag> type;
  };
\end{verbatim}
}

As a finishing touch, we will create an 
\textsf{object generator} for the transform iterator, which
 is a function that makes it more convenient to create objects of some
class template.

{\footnotesize
\begin{verbatim}
  template <class AdaptableUnaryFunction, class Iterator>
  typename transform_iterator_generator<AdaptableUnaryFunction,
      Iterator>::type
  make_transform_iterator(Iterator base,
      const AdaptableUnaryFunction& f = AdaptableUnaryFunction())
  {
      typedef typename transform_iterator_generator<AdaptableUnaryFunction,
        Iterator>::type result_t;
      return result_t(base, f);
  }
\end{verbatim}
}

Here is an example that shows how to use a transform iterator to
iterate through a range of numbers, multiplying each of them by 2
and printing the result to standard output.

{\footnotesize
\begin{verbatim}
  #include <functional>
  #include <algorithm>
  #include <iostream>
  #include <boost/iterator_adaptors.hpp>

  int main(int, char*[])
  {
    int x[] = { 1, 2, 3, 4, 5, 6, 7, 8 };
    const int N = sizeof(x)/sizeof(int);
    std::cout << "multiplying the array by 2:" << std::endl;
    std::copy(boost::make_transform_iterator(x, 
                std::bind1st(std::multiplies<int>(), 2)),
              boost::make_transform_iterator(x + N, 
                std::bind1st(std::multiplies<int>(), 2)),
              std::ostream_iterator<int>(std::cout, " "));
    std::cout << std::endl;
    return 0;
  }
\end{verbatim}
}

\noindent This output is: 
{\footnotesize
\begin{verbatim}
2 4 6 8 10 12 14 16
\end{verbatim}
}

\bibliographystyle{abbrv}
\bibliography{refs,tmpw00}

\end{document}
